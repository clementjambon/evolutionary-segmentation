\documentclass[11pt,a4paper]{article}

%page geometry
\usepackage[margin=0.7in]{geometry}

%encoding
%--------------------------------------
\usepackage[utf8]{inputenc} % input encoding
\usepackage[T1]{fontenc} % output encoding
%--------------------------------------

%French-specific commands
%--------------------------------------
\usepackage{babel}
\usepackage[autolanguage]{numprint}
%--------------------------------------

\usepackage{hyperref}

\usepackage{booktabs}

\usepackage{listings}
\usepackage{color}

\usepackage{graphicx}

\usepackage{caption}
\usepackage{subcaption}


\usepackage{amsmath}
\usepackage{amssymb}

\usepackage{algpseudocode}

\usepackage[shortlabels]{enumitem}

\newcommand{\cev}[1]{\reflectbox{\ensuremath{\vec{\reflectbox{\ensuremath{#1}}}}}}

\begin{document}

\title{Game Theory for Image Segmentation}
\author{Clément JAMBON \\ \href{mailto:cjambon@student.ethz.ch}{cjambon@student.ethz.ch}}
\maketitle

\begin{abstract}
    
\end{abstract}

\section{Introduction}

\section{Clustering game}

When evolutionary games involve a large number of possible strategies, it becomes hardly tractable to apply EGT as is and thus alternatives need to be considerated: Infection and Immunization Dynamics    

Interesting because as discussed in "Clustering Games" Marcello Pelillo and Samuel Rota Bul`, allows to move away from the standard "partitioning approach" which has two major limitations: 1. each object is mapped to one class 2. and only one (no soft assignment). Other method exist towards this goal (e.g. soft k-means TODO citation, deterministic annealing with Gibbs distribution TODO: citation )

With this framework, we do not need to know the number of clusters in advance.

Clustering games are non-cooperative games. "instead of two rational players, we allow competition between a large population of non-rational agents" "Note that, since every object is by definition strongly similar to itself, setting the diagonal of the payoff matrix to zero, or in general to a sufficiently low value, is of fundamental importance"

Section 3.10 for computational complexity

\section{Infection and Immunization Dynamics (InImDyn)}
Proceeds by injecting a strategy against which the current state is not immune through a selection strategy function. The main question is which selection strategy to adopt?

\section{Discussion}

Bulo introduced clustering games + would be interesting to extend them to higher-order similarities.

\end{document} 
